%for a more compact document, add the option openany to avoid
%starting all chapters on odd numbered pages
\documentclass[hidelinks, 12pt]{cmuthesis}

% This is a template for a CMU thesis.  It is 18 pages without any content :-)
% The source for this is pulled from a variety of sources and people.
% Here's a partial list of people who may or may have not contributed:
%
%        bnoble   = Brian Noble
%        caruana  = Rich Caruana
%        colohan  = Chris Colohan
%        jab      = Justin Boyan
%        josullvn = Joseph O'Sullivan
%        jrs      = Jonathan Shewchuk
%        kosak    = Corey Kosak
%        mjz      = Matt Zekauskas (mattz@cs)
%        pdinda   = Peter Dinda
%        pfr      = Patrick Riley
%        dkoes = David Koes (me)

% My main contribution is putting everything into a single class files and small
% template since I prefer this to some complicated sprawling directory tree with
% makefiles.

% some useful packages
\usepackage{times}
\usepackage{fullpage}
\usepackage{graphicx}
\usepackage{amsmath}
\usepackage{subfiles}
\usepackage[numbers,sort]{natbib}
\usepackage[backref,pageanchor=true,plainpages=false, pdfpagelabels, bookmarks,bookmarksnumbered,
%pdfborder=0 0 0,  %removes outlines around hyper links in online display
]{hyperref}
%% \usepackage{subfigure}

\usepackage{amsmath}
\usepackage{amssymb}
\usepackage{gensymb} 
\usepackage{dsfont}
\usepackage{graphicx}
\usepackage{float}
\let\labelindent\relax
\usepackage{enumitem}
\usepackage{hyperref}
\usepackage{tabu}
\usepackage{array}
\usepackage{caption}
\usepackage{subcaption}
\usepackage{colortbl}
\usepackage{tikz}
\usepackage{algorithm}
\usepackage{algpseudocode}
\usepackage{xparse}
\usetikzlibrary{positioning}


\DeclareDocumentCommand{\particles}{ O{} O{} O{}}{^{#2}X_{#1}^{#3}}
% \particle[time][index][section]
\DeclareDocumentCommand{\particle}{ O{} O{} O{}}{^{#2}x_{#1}^{#3}}
\newcommand{\state}{x}

\newcommand{\sampled}{\tilde}
\newcommand{\subcap}[1]{\textbf{(\subref{#1})}}
\newcommand{\maction}{\mathcal{M}}
%% \newcommand{\particles}{\mathcal{P}}
%% \newcommand{\particle}{x}
\newcommand{\bin}{b}
\newcommand{\groups}{L}
\newcommand{\measurement}{m}
\newcommand{\measurementSet}{M}
\newcommand{\totalWeight}{\mathcal{W}}
\newcommand{\feature}{\mathcal{S}}
\renewcommand{\algorithmicrequire}{\textbf{Input:}}
\renewcommand{\algorithmicensure}{\textbf{Output:}}
\def\BState{\State\hskip-\ALG@thistlm}

% Approximately 1" margins, more space on binding side
%\usepackage[letterpaper,twoside,vscale=.8,hscale=.75,nomarginpar]{geometry}
%for general printing (not binding)
\usepackage[letterpaper,twoside,vscale=.8,hscale=.75,nomarginpar,hmarginratio=1:1]{geometry}

% Provides a draft mark at the top of the document. 
\draftstamp{\today}{DRAFT}

\begin {document} 
\frontmatter

%initialize page style, so contents come out right (see bot) -mjz
\pagestyle{empty}

\title{ %% {\it \huge Thesis Proposal}\\
{\bf Planning and Localization Using Contacts}}
\author{Bradley L. Saund}
\date{April 2017}
\Year{2017}
\trnumber{}

\committee{
Howie Choset, Co-chair \\
Reid Simmons, Co-chair \\
Matt Mason \\
Arun Srivatsan
}

\support{}
\disclaimer{}

% copyright notice generated automatically from Year and author.
% permission added if \permission{} given.

%% \keywords{}

\maketitle

%% \begin{dedication}

%% \end{dedication}

\pagestyle{plain} % for toc, was empty

%% Obviously, it's probably a good idea to break the various sections of your thesis
%% into different files and input them into this file...

\begin{abstract}
  Contacting the world can provide stability, support, and sensory information, however many robots avoid contact whenever possible.
  Contacts present the physical danger of large forces as well as challenging computational issues, but enabling robots to reason about contacts extends the extent to which robots can perceive and act in the world.

  This thesis explores both localization and planning where contacts are required.
  Computational issues arise due to the local nature of contacts, yielding subspaces of intersection thin manifolds, so several common techniques in robotics are adapted to better handle the local effects of contact
  A particle filter is augmented to update from a precise measurement that would ordinarily eliminate most particles.
  An efficient information gain metric is defined using these particles to predict useful future measurements.
  A new dynamics model and associated cost function are created for a robotic arm, which although are not faithful to real dynamics, are conditioned well for use in existing planning methods.
  This artificial but useful dynamics model is shown in both trajectory optimization and sampled-based planning.
  The techniques are demonstrated in simulation and physical hardware.
\end{abstract}

\begin{acknowledgments}
  This thesis and my work in graduate school would not be possible without the collaboration and guidance of many intelligent people.
  First, I would like to thank my advisors. Both Howie and Reid have provided me with support and guidance, and most importantly have forced me to justify my theories and guided me to many new concepts in robotics. Their advising styles were different, but worked amazingly well together, and I am a smarter person thanks to their influence.
  
  I would like to thank the senior PhD students who I sat near and from whom I absorbed knowledge. Discussions with Glenn, Arun, and Chao frequently inspired me. In addition, discussions and help from Guillaume, Alex, Matt, Julian, Ky, Elena, Breelyn, and many others have both directly influenced my work and generally improved my robotics knowledge.
  The former graduate students now at Hebi designed and provided the hardware that first inspired my snake arm planning goals and they have continued to support and improve the hardware.

  I also would like to thank Shiyuan Chen, which whom I have extensively collaborated over the course of this thesis and whose work was heavily used in the localization section of this thesis.

  I owe so much to my parents for providing me with a scientifically rich childhood, always encouraging me to excel.
  
  And of course I would like to thank the love of my life, Katie Brennan, for agreeing to two years of long distance during this masters program.
  
\end{acknowledgments}



\tableofcontents
\listoffigures
%% \listoftables

\mainmatter

%% Double space document for easy review:
%\renewcommand{\baselinestretch}{1.66}\normalsize

% The other requirements Catherine has:
%
%  - avoid large margins.  She wants the thesis to use fewer pages, 
%    especially if it requires colour printing.
%
%  - The thesis should be formatted for double-sided printing.  This
%    means that all chapters, acknowledgements, table of contents, etc.
%    should start on odd numbered (right facing) pages.
%
%  - You need to use the department standard tech report title page.  I
%    have tried to ensure that the title page here conforms to this
%    standard.
%
%  - Use a nice serif font, such as Times Roman.  Sans serif looks bad.
%
% Other than that, just make it look good...


\chapter{Introduction}
\subfile{Introduction/intro.tex}

\chapter{Related Work} \label{chap:related_work}
\subfile{RelatedWork/relatedWork.tex}

\chapter{Localization using Contacts} \label{chap:localization}
\subfile{Localization/localization.tex}

\chapter{Planning with Contacts for Support} \label{chap:planning}
\subfile{Planning/planning.tex}

\chapter{Conclusion and Future Work}
\subfile{Conclusion/conclusion.tex}

%\appendix
%\include{appendix}

\backmatter

%\renewcommand{\baselinestretch}{1.0}\normalsize

% By default \bibsection is \chapter*, but we really want this to show
% up in the table of contents and pdf bookmarks.
\renewcommand{\bibsection}{\chapter{\bibname}}
%\newcommand{\bibpreamble}{This text goes between the ``Bibliography''
%  header and the actual list of references}
\bibliographystyle{plainnat}
\bibliography{FineLocalizationBib,DatumBib,nsfBib,BracingBib,additional} %your bib file
\nocite{*}

\end{document}
