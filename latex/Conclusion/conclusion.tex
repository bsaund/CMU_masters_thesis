\documentclass[../thesis.tex]{subfiles}

\begin{document}



This paper introduced the \textit{datum-based particle filter}, which provides a method to localize a task location defined by datums on an object with internal degrees of freedom.
This method stores the belief as the belief over the poses of multiple rigid sections comprising the object using a particle filter, and selects measurement actions using the metric of information gain.
Two implementations are described: a high dimensional particle filter capturing the full state, and multiple particle filters coupled through the tolerances between sections.
The techniques to avoid particle starvation during rigid body localization have been extended to both implementations of the datum-based particle filter. 
Information gain of a potential measurement action is approximated as a discrete probabilistic decision process over the particles comprising the belief.
The formulation presented distinguishes between useful information which updates the belief of the target feature, and information which only improves the belief of non-datum sections.


\end{document}
