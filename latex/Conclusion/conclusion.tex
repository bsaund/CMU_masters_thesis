\documentclass[../thesis.tex]{subfiles}

\begin{document}



%% This paper introduced the \textit{datum-based particle filter}, which provides a method to localize a task location defined by datums on an object with internal degrees of freedom.
%% This method stores the belief as the belief over the poses of multiple rigid sections comprising the object using a particle filter, and selects measurement actions using the metric of information gain.
%% Two implementations are described: a high dimensional particle filter capturing the full state, and multiple particle filters coupled through the tolerances between sections.
%% The techniques to avoid particle starvation during rigid body localization have been extended to both implementations of the datum-based particle filter. 
%% Information gain of a potential measurement action is approximated as a discrete probabilistic decision process over the particles comprising the belief.
%% The formulation presented distinguishes between useful information which updates the belief of the target feature, and information which only improves the belief of non-datum sections.

This thesis has explored the challenges and benefits of contacts in two robotics problems.
Contacts can provide accurate sensory information and aid in robot motion, but the local nature of contact forces adds difficulties.
Specifically considered were a localization problem using a touch probe, and a planning problem using contacts for support.
%% Chapter \ref{chap:related_work} reviewed existing methods for localization and planning using contacts.

In localization, touch probing yields accurate measurements about the location of the surface of a part.
However, only a thin manifold of configurations are consistent with the measurement, which presents challenges when using this measurement to update the belief of the object.
Chapter \ref{chap:localization} first reviewed work done in collaboration with Shiyuan Chen \cite{Saund2017} on designing a particle filter capable of updating using contact measurement.
It then extends this particle filter to parts with internal degrees of freedom, where the goal is to localize a specific section defined by datums.
This chapter describes how to use the particles to define an information gain metric that can be calculated efficiently, and this metric is used to select informative probing actions during experiments.

In motion planning, contact forces can reduce joint torques in a cantilevered robotic arm.
However, utilizing contacts involves discovering and traversing a thin manifold through configuration space, which presents difficulties for common methods of planning.
Chapter \ref{chap:planning} describes approaches for trajectory optimization and samples based planners for handling the effects of contacts.
The heart of the approach is a dynamics model that can provide unrealistically optimistic contact forces at a distance, and thereby providing useful gradient information to a cost function.
The degree to which forces at a distance can be used is controllable, enabling the enforcement of realistic final trajectories.
However, an unfortunate characteristic of this method is that all constraints are soft, thus tuning of hyperparameters can be needed to achieve acceptable solutions.


There is a plethora of future directions for this work.
Perhaps most obviously, these two approaches can be combined on the same robot.
The same contacts used for planning can also be used for localization, augmenting the planning cost function further to encourage informative contact locations.
%% Progress can still be made independently.

This thesis demonstrated the localization approach on only a single physical part, and only a limited set in simulation.
The datum based particle filter can be applied to a wide variety of environments consisting of objects with Bayesian linked poses.
Its power is the ability to focus on only the information gathering actions that aid a task, ignoring uncertainty when acceptable.
Through collaborators we have identified a practical application in need of the approach described involving a robot in a manufacturing setting.

Obvious extensions exist to the planning methods described. The sample based planner described will yield only feasible trajectories, not optimal trajectories, but the methods of RRT* can be used to produce better trajectories for initializing the optimization.
In addition, when the trajectory optimization fails the current solution is a random restart, however this new optimization may result in convergence to the same local minima.
The same approaches employed in section \ref{sec:sample_planning} can be used to avoid repeated exploration of the same region.
Finally, the dynamics model and cost function used were constructed specifically for contacts, but the methods used may be extended to other challenges in robotics.



\end{document}
